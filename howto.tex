\documentclass[DM]{lsstdoc}

\author{G. Comoretto}


\begin{document}

\date{\today}

\title{How To Execute docsteady}
\mkshorttitle

\section{Introduction}

This is intended to be a short guide to help the installation and  execution of docsteady.


\section{Installation}\label{sec:install}

Create a conda environment based on docsteady:

\texttt{conda create --name docsteady-env docsteady -c gcomoretto}

Ensure that the conda configuration file  \texttt{.condarc} does not include the conda-forge channel.

To use docsteady, activate the environment as follows:

\texttt{conda activate docsteady-dev}

This environment will provide all dependencies that are required to build and run docsteady.

It is recommended to use the provided conda environment also for development activities, see section \ref{sec:development}.

Next release will make the distribution compatible with conda-forge. Future release should be provided in conda-forge channel instead of a private channel.


\section{Execution}

The information provided in this section is also available using the \texttt{--help} option.

In addition, existing documents have a \texttt{.docugen} file with the command to be executed in CI, that can be used as an example.

It is recommended to setup the following environment variables:

\begin{itemize}
\item JIRA\_USER
\item JIRA\_PASSWORD
\end{itemize}

otherwise it is required to specify them from command line.


\subsection{VCD genration}

the VCD is generating extracting information from the Jira Database.
Since the access to the Jira database is possible only from the Tucson network, it is required to be connected via VPN.

\texttt{docsteady generate-vcd --sql True DM jira\_docugen.tex}

The command extracts from Jira all information regarding \textbf{DM} and generate the file \texttt{jira\_docugen.tex}.
This file is meant to be included in LDM-639.tex.

The VCD is generated accessing the database instead of using the REST API. Therefore the following environment variables should be specified:

\begin{itemize}
\item JIRA\_VCD\_USER
\item JIRA\_VCD\_PASSWORD
\end{itemize}


\subsection{Test Specification generation}


\subsection{Test Plan and Report generation}

\texttt{docsteady generate-tpr LVV-P72 DMTR-231.tex --trace true}

The command extract the information from Jira using REST API.
Each test plan and report has to correspond to a Jira test plan (i.e. LVV-P72).
The information is stored in a tex file, i.e. \texttt{DMTR-231.text}, that can be build directly.
An appendix file is also created, in this case \texttt{DMTR-231.appenx.tex}.

However additional files is required.

\begin{itemize}
\item Makefile
\item appendix.tex
\item history\_and\_info.tex
\end{itemize}

It should be possible to have the repository setup by sqrbot-jr (not available yet).


\subsection{Verification Element Baseline generation}


\section{Development}
\label{sec:development}

Despite Docsteady is a pure python tool, it depends on \textbf{pandoc} that is a \texttt{c++} compiled library available only as conda package.

Therefore, in order to fix the environment and ensure the expected behavior, it is important to set-up a consistent conda environment to use for development.
The environment set-up is explained in section \ref{sec:install}.

The source code of available at \url{https://github.com/lsst-dm/docsteady}.

To test changes done locally in the source code, use the following procedure:

\begin{itemize}
\item (if not available) create the environment as specified in \ref{sec:install}
\item activate the environment \texttt{conda activate docsteady-env}
\item do your changes
\item install the updates in the docsteady-dev environment: \texttt{python setup.py install}
\item activate docteady-dev in a different terminal to update an existing document or to render a new one
\end{itemize}

Each time a change is done in source code, in order to make it available in the environment, the following command has to be executed:

\texttt{python setup.py install}

from the docsteady root folder.


\section{Documentation Procedure}
\label{sec:docproc}

Documents autogeneration may become very confusing if not done in a programatic way.
Please consider the DM documentation approach as a guideline, summarized here.


\end{document}
