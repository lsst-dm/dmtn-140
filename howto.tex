\documentclass[DM]{lsstdoc}

\author{G. Comoretto}


\begin{document}

\date{\today}

\title{How To Execute docsteady}
\mkshorttitle

\section{Introduction}

This is intended to be a short guide to help the installation and  execution of docsteady.


\section{Installation}\label{sec:install}

Create a conda environment based on docsteady:

\texttt{conda create --name docsteady-env docsteady -c gcomoretto}

Ensure that the conda configuration file  \texttt{.condarc} does not include the conda-forge channel.

To use docsteady, activate the environment as follows:

\texttt{conda activate docsteady-env}

This environment will provide all dependencies that are required to build and run docsteady.

It is recommended to use the provided conda environment also for development activities, see section \ref{sec:development}.

Future release should be provided in conda-forge channel instead of a private channel.



\section{Execution}

The information provided in this section is also available using the \texttt{--help} option.

In addition, existing documents have a \texttt{.docugen} file with the command to be executed in CI, that can be used as an example.

It is recommended to setup the following environment variables:

\begin{itemize}
\item JIRA\_USER
\item JIRA\_PASSWORD
\end{itemize}

otherwise it is required to specify them from command line.

In order to execute the tool, with any of the syntaxes described in the following subsections, the corresponding conda environment providing docsteady shall be activated.



\subsection{VCD genration}

A VCD is generating extracting information from the Jira Database.
Since the access to the Jira database is possible only from the Tucson network, it is required to be connected via VPN.

The following command extracts from Jira all information regarding \textbf{DM} and generate the file \texttt{jira\_docugen.tex}:

\texttt{docsteady generate-vcd --sql True DM jira\_docugen.tex}

This file is meant to be included in LDM-639.tex.
This syntax assumes that the environment variables with the credentials to access the database have been exported in the shell where the command is executed.

The VCD is generated accessing the database instead of using the REST API. Therefore the following environment variables should be specified:

\begin{itemize}
\item JIRA\_VCD\_USER
\item JIRA\_VCD\_PASSWORD
\end{itemize}

In case you want to generate VCD for a different LSST/Rubin Observatory subsystem, for example the \textit{Camera} subsystem,
just use the corresponding code configure in Jira, CAM, instead of DM.

The subsystems configured in the field \textit{Component} in Jira (October 2020) are:
\begin{itemize}
\item DM
\item EPO
\item OCS
\item PSE
\item SAF
\item TS (T\&S)
\end{itemize}



\subsection{Test Specification Generation}

A test specification is extracted using REST API.
All test included in a TM4J folder, including subfolders, are rendered in the same extraction.
The folder organization in Jira shall correspond to the major components.

The syntax to extract a test specification information is the following:

\texttt{docsteady generate-spec "</tm4j\_folder>" jira\_docugen.tex}

where \textit{</tm4j\_folder>} shall be replaced by the exact folder where the test cases are defined in Jira, textit{Test Cases} tab.
For example, the command to extract the DM Acceptance test specification, LDM-639, is the following:

\texttt{generate-spec "/Data Management/Acceptance|LDM-639" jira\_docugen.tex}

Note that:
\begin{itemize}
\item the output \textit{jira\_docugen.tex} has to be included in the LDM-639.tex.
\item the folder name in Jira, includes at the end the test specification identification code. This helps orienting in the folder structure
\item an appendix with the traceability to the requirements is produced, \textit{jira\_docugen.appendix.tex}, to be included in included in the test specification.
\end{itemize}



\subsection{Test Plan and Report Generation}

\texttt{docsteady generate-tpr LVV-P72 DMTR-231.tex --trace true}

The command extract the information from Jira using REST API.
Each test plan and report has to correspond to a Jira test plan (i.e. LVV-P72).
The information is stored in a tex file, i.e. \texttt{DMTR-231.text}, that can be build directly.
With the option: \textbf{--trace true}, an appendix file is also created, in this case \texttt{DMTR-231.appenx.tex}.

However additional files is required.

\begin{itemize}
\item Makefile
\item appendix.tex
\item history\_and\_info.tex
\end{itemize}

When creating the git repository using \textbf{sqrbot-jr}, all the requested files should already be present.

In case you want to generate a test report for a different subsystem, you can specify the  namespace:

\texttt{docsteady --namespace se generate-tpr LVV-P63 SCTR-14.tex}



\subsection{Verification Element Baseline generation}

Verification Elements (VE) are Jira issues in the LVV Jira project, of type \texttt{Verification}.
They are categorized in Components (DM, SITCOM, etc) and Sub-Componenst.

A VE baseline document is extracted using REST API.
All VE associated with a Jira Component or Sub-Component, if specified, are rendered in the same extraction.

The syntax to extract a VE baseline information is the following:

\texttt{docsteady baseline-ve <COMPONENT> <SUB-COMPONENT> jira_docugen.tex [--details true]}

The informatino is saved in a file colled \texttt{jira_docugen.tex}. You may choose a different name.
This file has to be included in a Latex document, where the corresponding information about the Component and Sub-Component is provided.

If the option \texttt{--details true} is provided, an extra technical note is generated, with including all test case details.


\subsubsection{Sub-Components}

Ideally, Sub-Comonents are matted to the major products of an organization. 
The shall also be mapped to the product tree defined in the model.

In DM, trying to find a good balance between details and practicity the following components have been defined:

\begin{itemize}
\item Science
\item Service
\item Network
\item Infrastructure
\end{itemize}

For each of these subcomponents, a different VE baseline document is extracted.



\section{Development}
\label{sec:development}

Despite Docsteady is a pure python tool, it depends on \textbf{pandoc} that is a \texttt{c++} compiled library available only as conda package.

Therefore, in order to fix the environment and ensure the expected behavior, it is important to set-up a consistent conda environment to use for development.
The environment set-up is explained in section \ref{sec:install}.

The source code of available at \url{https://github.com/lsst-dm/docsteady}.

To test changes done locally in the source code, use the following procedure:

\begin{itemize}
\item (if not available) create the environment as specified in \ref{sec:install}
\item activate the environment \texttt{conda activate docsteady-env}
\item do your changes
\item install the updates in the docsteady-dev environment: \texttt{python setup.py install}
\item activate docteady-dev in a different terminal to update an existing document or to render a new one
\end{itemize}

Each time a change is done in source code, in order to make it available in the environment, the following command has to be executed:

\texttt{python setup.py install}

from the docsteady root folder.


\section{Documentation Procedure}
\label{sec:docproc}

Documents autogeneration may become very confusing if not done in a programatic way.
Please consider the DM documentation approach as a guideline, summarized here.

\begin{itemize}
\item Create an handler in DocuShare
\item Create a repository in Github (using sqrbot-jr)
\item Configure the continuous integrationas described in next Sec.~\ref{sec:ci}.
\item Never auto-generate the document directly to master
\item Render the document to a ticket branch, or to the \textbf{jira-sync} special branch
\item Ensure that the document is correctly publish in the corresponding LSST The Docs landing page and that everybody who is interested in it can access.
\item When a set of activities are completed, merge the branch to master using a PR. The R can be used to let contributors and stakeholders comment to the changes before merge.
\item In case the special \textbf{jira-sync} branch is used, after merging it to master, I suggest to delete it  and recreate from the latest master.
\end{itemize}



\section{Continuous Integration}\label{sec:ci}

The real added value of this tooling, is the capability to auto-generate continuously the documents from Jira.
This is done using the Jenkins service available at:

\url{https://lsst-docs-ci.ncsa.illinois.edu/jenkins/}

This service is at the moment behind NCSA firewall, and therefore it requires VPN to access and monitor the jobs.
Authentication to this Jenkins instance is manage via OATH.
To permit a user to access the repository, it need to have access as a member to the \texttt{docs-ci} GitHub organization.

The rendered documents are available in the corresponding GitHub repositories and LSST The Docs landing pages.

The docugen jobs are create using the seeds script defined at the following location:

\url{https://github.com/docs-ci/docs-ci-seeds/blob/master/jobs/docugen_jobs.groovy}

and the jobs to be configured are defined here:

\url:{https://github.com/docs-ci/docs-ci-seeds/blob/master/etc/docugen.yaml}

using a simple yaml file logic.

Since the \textit{lsst-docs-ci} service is behind a firewall, it the seeds job: 

\url{https://lsst-docs-ci.ncsa.illinois.edu/jenkins/job/Service/job/seeding/}

cannot be triggered just pushing to the seeds repository, but it requires to be executed manually.



\end{document}
