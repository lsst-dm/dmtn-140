
\begin{abstract}
LSST Data Management (DM) is responsible for creating the software, services, and systems that will be used to produce LSST's science-ready data products. The software, currently under development is heterogeneous, comprising both C++ and Python components, and is designed to facilitate both the processing of LSST images and to enable value-added contributions from the broader scientific community. Verification and validation of these software products, services, and systems is an essential yet time-consuming task. In this paper, we present the tooling and procedures used by the LSST project for a systematic verification and validation documentation approach. By adopting a systematic approach, we guarantee full traceability to system requirements, integration with the project’s System Engineering model, and substantially reduce the time required for the whole process.
As the Commissioning Execution Plan (LSE-390) says, "The project team shall
deliver all reports documenting the as-built hardware and software including:
drawings, source code, modifications, compliance exceptions, and recommendations
for improvement." As a first step towards the delivery of documents that will describe the system at the
end of construction, we are assembling teams for producing of the order 40 papers
that eventually will be submitted to relevant professional journals. The immediate goal is to accomplish
all the writing that can be done without data analysis before the data
taking begins, and the team becomes much more busy and stressed.

This document provides the template for these papers.
\end{abstract}

